\section{Konklusion}
Drivhus styringen som er blevet implementeret viser, 
at der kan findes en løsning med komponenter som ikke er beregnet til at kommuniker sammen.
I dette tilfælde IHC fugt- og temperatur sensoren brugt inde i drivhus hvorefter dens værdi,
bliver kommuniker videre til styringen. En bedre løsning ville være, at bruge en anden sensor
som er designet til det høje luftigheds miljø. 
\\ \\
Grunden til, at vælge en intelligent bygningsinstallation er at øge ens comfort. 
At få automatiseret blandt andet lyset efter funktionen af rummet, kører forceret drift på ventilationsanlægget
når man ikke er hjemme. 
Eller bruge en fugt sensor i på badværelset, hvis den relative fugt niveau kommer over de 30\% at ventilationsanlægget køre op i hastighed.
Der bruges en IHC controller til lysstyringen, og der ønskes at ventilationsanlægget kan styres igennem IHC.

Det betyder, at i valgt af ventilationsanlægget skal der tages højde for at dens styring kan overstyres af en ekstern enhed.
Ventilationsproducent Nilan har lavet en styring (CTS 602) som har en funktionsblok i IHC, 
hvilket gøre det mere simple at få kommunikationen til at virke mellem de to styringer.
I tilbuddet er der valgt et Nilan Comfort 600 med en CTS 602 styring, så integrationen i IHC styring er dokumenteret fra producentes side.


