\subsection{Ventilation} \label{sub:ventilation}
Der tages udgangspunkt i BR18 \cite{BR18:Online} kravene.
Under \S443 står kravene beskrevet at der skal være en udelufttilførsel på mindst $0,3 l/s pr. m^2$.
I \S443 stk. 3 står der at udsugningen i køkkener skal forøges til mindst $20 l/s$ og \S443 stk. 4 beskriver udsugningen fra bade- og wc-rum skal kunne forøges til mindst 15 l/s.
Desuden beskreves der også at wc-rum uden bad og bryggers skal der kunne udsuges mindst $10 l/s$. 
I forhold til BR15 så der ikke nogle ændring i kravene som har betydning for beregningen.
Dog beskriver BR18, i \S444 at kælder i enfamileshus skal der udsuges mindst $10 l/s$.

\subsubsection{Bestemmelse af minimumskravet til anlægget} \label{subsub:minimumkrav_ventilation}
Da kravene for lufttilførelsen kun gælder for de opvarmet beboelse kvm,
det betyder at ydre væggeikke behøves at være med i udregningen.
Ud fra det aspekt har der kunne konstateret, at kvm målene på tegningen er inklusiv væggene.
For at få en mere præcis beregning af hvad ventilationssystemet skal kunne levere i til indblæsning,
er der målt op på plantegningen og derefter udregnet kvm. \\
Beregningseksemple og tabel over beregningerne for indblæsning er i \nameref{sub:indblaesning_beregning} på side \pageref{sub:indblaesning_beregning}. \\
Beregningseksemple og tabel over beregningerne for udsugning er i \nameref{sub:udsugning_beregning} på side \pageref{sub:udsugning_beregning}.

\begin{center}
    $
    d_{n} = \sqrt{ \frac{q_{v} \cdot 4}{V\cdot\pi\cdot3600}}
    $ \\ 
\end{center}

Hvor $d_n$ er diameteren i meter, $V$ er lufthastigheden i $m/s$ og $q_v$ er volumenstrømmen i $m^{3}/t$.

\subsubsection{Indblæsning}
\label{subsub:indblaesning}

\subsubsection{Udsugning}
\label{subsub:udsugning}