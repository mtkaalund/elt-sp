\subsection{Ventilation} \label{sub:ventilation}
Der tages udgangspunkt i BR18 [\cite{BR18:Online}] kravene.
Under \S443 står kravene beskrevet at der skal være en udelufttilførsel på mindst 0,3 l/s pr. m$^2$.
I \S443 stk. 3 står der at udsugningen i køkkener skal forøges til mindst 20 l/s og \S443 stk. 4 beskriver udsugningen fra bade- og wc-rum skal kunne forøges til mindst 15 l/s.
Desuden beskreves der også at wc-rum uden bad og bryggers skal der kunne udsuges mindst 10 l/s. 
I forhold til BR15 så der ikke nogle ændring i kravene som har betydning for beregningen.
Dog beskriver BR18, i \S444 at kælder i enfamileshus skal der udsuges mindst 10 l/s.

\subsubsection{Bestemmelse af minimumskravet til anlægget} \label{subsub:minimumkrav_ventilation}
Da kravene for lufttilførelsen kun gælder for de opvarmet beboelse kvm,
det betyder at ydre vægge ikke behøves at være med i udregningen.
I BR18 [\cite{BR18:Online}], 
så skal udelufttilførsel minimum være med 0,3 l/s pr. m$^2$ opvarmet etageareal.
For at kunne lave beregningen, så skal det udregnes til en volumenstrømmen som er i m$^3$/t. 
Dette gøres med at gange 3,6 $\frac{ \text{m}^3 \text{/ l} }{\text{ t / s} }$ for at omregne fra sekunder til timer og liter til kubicmeter.
i tabel \ref{table:samregn_vent_ind} på side \pageref{table:samregn_vent_ind} har jeg regnet ud fra rummenes kvm.

\begin{table}[!h]
     \begin{center}
        \begin{tabular}{|l|r|r|r|}
            \hline
            Rum & kvm & Udelufttilførsel i l/s & Udelufttilførsel i m$^3$/t \\
            \hline
            Soveværelse       & 12,6 m$^2$ & 3,78 l/s & 13,61 m$^3$/t\\
            Stue / Alrum / Køkken & 39,7 m$^2$ & 11,91 l/s & 42,88 m$^3$/t\\
            Entré / Bryggers  & 9,6 m$^2$ & 2,88 l/s & 10,37 m$^3$/t\\
            Gang              & 2,3 m$^2$ & 0,69 l/s & 2,48 m$^3$/t\\
            Værelse 1         & 11,9 m$^2$ & 3,57 l/s & 12,85 m$^3$/t\\
            Værelse 2         & 12,0 m$^2$ & 3,60 l/s & 12,96 m$^3$/t\\
            Bad 1             & 4,3 m$^2$ & 1,29 l/s & 4,64 m$^3$/t\\
            Bad 2             & 7,5 m$^2$ & 2,29 l/s & 8,24 m$^3$/t\\
            \hline
            \hline
            Samlet & \underline{99,9 m$^2$} & \underline{30,01 l/s} & \underline{108,03 m$^3$/t} \\
            \hline
        \end{tabular}
    \end{center}
    \caption{Udregning af Udelufttilførelsens minimumskrav}
    \label{table:samregn_vent_ind}
\end{table}

Så vores krav til udelufttilførsel er at anlægget skal kunne minimum lave et luftskifte på 30,01 l/s.
I BR18 [\cite{BR18:Online}] beskreves der også hvor meget luft som skal udsuges i udvalgte rum, 
de tal skal ligges sammen og ud fra det kan laves en sammenregning til minimumskravet. De udregning findes i tabel \ref{table:samregn_vent_ud} på side \pageref{table:samregn_vent_ud}.
\begin{table}[!h]
    \begin{center}
       \begin{tabular}{|l|r|r|}
           \hline
           Rum & Udsugning i l/s & Udsugning i m$^3$/t \\
           \hline
           Stue / Alrum / Køkken & 20 l/s & 72 m$^3$/t\\
           Entré / Bryggers  & 15 l/s & 54 m$^3$/t\\
           Bad 1             & 15 l/s & 54 m$^3$/t\\
           Bad 2             & 15 l/s & 54 m$^3$/t\\
           \hline
           \hline
           Samlet & \underline{65 l/s} & \underline{234 m$^3$/t} \\
           \hline
       \end{tabular}
   \end{center}
   \caption{Udregning af udsugnings minimumskrav}
   \label{table:samregn_vent_ud}
\end{table}
Her kan det ses, at minimumskrav til udsugning er højere end til udelufttilførselen. 
Da der ønskes, at der er en ligevægt i ens ventilationssystem så beregnes rør diameteren udfra minimum volumenstrømmen i udsugningen.
\begin{equation}\label{eqn:udregning_rd}
d_{n} = \sqrt{ \frac{q_{v} \cdot 4}{V\cdot\pi\cdot3600}}
\end{equation}
Ligning (\ref{eqn:udregning_rd}) på side \pageref{eqn:udregning_rd} bruges til, udregne diameteren i meter, $V$ er lufthastigheden i $m/s$ og $q_v$ er volumenstrømmen i $m^{3}/t$.
Typisk er lufthastigheden mellem 4 - 10 m/s. 
\begin{align} \label{eqn:udregning_min_rd} 
    d_{n}       &= \sqrt{ \frac{q_{v} \cdot 4}{V\cdot\pi\cdot3600}} = \sqrt{ \frac{234 \cdot 4}{4\cdot\pi\cdot3600}} = 0,1438 m = 143,8 mm
\end{align}
Beregningen af rørdiameteren i ligning (\ref{eqn:udregning_min_rd}) på side \pageref{eqn:udregning_min_rd} er fundet til 143,8 mm, 
ikke findes i lindab's sortiment så vælges rørdiameteren til 160mm.

\subsubsection{Tryktab i udsugning} \label{subsub:tryktab_udsugning}
Der er en oversigts tegning over ventilationssystem på side \pageref{fig:tegning_ventr}. 
Længderne er skrevet på tabel \ref{table:oversigt_l_udsugning} på side \pageref{table:oversigt_l_udsugning}, 
der er også valgt at inkludere hvilket volumenstrømmen som skal være i de enkelte rør. 
\begin{align} \label{eqn:volumenstroem_sammenregning} 
    q_{v}(E) &= q_{v}(F) + q_{v}(H) = 54 + 54 = 108 \text{ m}^3\text{/t}
\end{align}
Alle volumenstrømmene skal omregnes til lufthastighed, 
for at kunne finde frem til tryk tabet i rørene, 
det gøres i ligningen \ref{eqn:omregning_vs_til_lh}.
\begin{align} \label{eqn:omregning_vs_til_lh}
    d_{n} &= \sqrt{ \frac{ q_v \cdot 4 }{ V\cdot \pi \cdot 3600 } }  \nonumber \\
          &\Downarrow  \nonumber \\
    d_{n}^{2} &= \frac{ q_v \cdot 4 }{ V\cdot \pi \cdot 3600 } \nonumber \\
          &\Downarrow \nonumber \\
    V     &= \frac{ q_v \cdot 4 }{ d_{n}^{2} \cdot \pi \cdot 3600 } 
\end{align}
Ved at bruge formulen i \ref{eqn:omregning_vs_til_lh}, 
giver den lufthastighed som bruges til, 
at finde tryktabet per meter i databladet for røret.
\begin{table}[h!]
    \begin{center}
       \begin{tabular}{|l|r|r|r|r|r|}
           \hline
           Længde & meter & q$_{v}$ & V & P$_{a}$ / m & Tryktab\\
           \hline
           A & 1,38 m & 270 m$^3$/t & 3,73 m/s & 1,2 P$_{a}$/m & 1,656 Pa\\
           B & 0,27 m & 162 m$^3$/t & 2,24 m/s & 0,5 P$_{a}$/m & 0,135 Pa\\
           C & 0,75 m & 72  m$^3$/t & 0,99 m/s & 0,1 P$_{a}$/m & 0,075 Pa\\
           D & 4,43 m & 54  m$^3$/t & 0,75 m/s & 0,1 P$_{a}$/m & 0,443 Pa\\
           E & 0,46 m & 108 m$^3$/t & 1,49 m/s & 0,2 P$_{a}$/m  & 0,092 Pa\\
           F & 0,94 m & 54 m$^3$/t & 0,75 m/s & 0,1 P$_{a}$/m & 0,094 Pa\\
           G & 2,79 m & 54 m$^3$/t & 0,75 m/s & 0,1 P$_{a}$/m & 0,279 Pa\\
           H & 0,45 m & 54 m$^3$/t & 0,75 m/s& 0,1 P$_{a}$/m & 0,045 Pa\\
           \hline
       \end{tabular}
   \end{center}
   \caption{Oversigt over længderne brugt i udsugningen}
   \label{table:oversigt_l_udsugning}
\end{table}
I tabel \ref{table:oversigt_l_udsugning}, er lufthastigheden som skal bruges til at finde tryktabet i T-rørene.
Dette gøres ved, at aflæse graferne i databladet for T-stykkerne.
\begin{table}[h!]
    \begin{center}
       \begin{tabular}{|l|r|r|r|}
           \hline
           T-rør & V$_{1}$ & V$_{2}$ & Tryktab \\
           \hline
           T$_{\text{B->A}}$ & 2,24 m/s & 3,73 m/s & 4,8 Pa \\ 
           T$_{\text{C->A}}$ & 0,99 m/s & 3,73 m/s & 2,8 Pa \\
           T$_{\text{E->B}}$ & 1,49 m/s & 2,24 m/s & 0,6 Pa \\
           T$_{\text{D->B}}$ & 0,75 m/s & 2,24 m/s & 1,5 Pa \\
           T$_{\text{F->E}}$ & 0,75 m/s & 1,49 m/s & 0,8 Pa \\
           T$_{\text{G->E}}$ & 0,75 m/s & 1,49 m/s & 1,0 Pa \\
           \hline
       \end{tabular}
   \end{center}
   \caption{Oversigt tryktabet i T-rør}
   \label{table:oversigt_tryktab_t-roer}
\end{table}
Nu kan tryktabet findes på det stykke som er længes væk fra ventilationsanlægget, 
som er rum bad 2.
\begin{table}[h!]
    \begin{center}
       \begin{tabular}{lcr}
           \hline
           \hline
           \textbf{Bad 2} &  & \\
           \hline
           \hline
           Ventil (-5mm åbning) & : & 21,000 Pa \\
           90$^\circ$ BU    & : & 0,500 Pa \\
           Rør$_{\text{H}}$ & : & 0,045 Pa \\
           90$^\circ$ BU    & : & 0,500 Pa \\
           Rør$_{\text{G}}$ & : & 0,279 Pa \\
           T-Stykke$_{\text{G->E}}$  & : & 1,000 Pa\\
           Rør$_{\text{E}}$ & : & 0,092 Pa \\
           T-Stykke$_{\text{E->B}}$  & : & 0,600 Pa\\
           Rør$_{\text{B}}$ & : & 0,135 Pa \\
           T-Stykke$_{\text{B->A}}$  & : & 4,800 Pa\\
           Rør$_{\text{A}}$ & : & 1,656 Pa \\
           \hline
           Samlet tryktab    & : & \underline{\underline{ 30,607 Pa}} 
       \end{tabular}
   \end{center}
   %\caption{Oversigt tryktabet i T-rør}
   %\label{table:oversigt_tryktab_t-roer}
\end{table}

\subsubsection{Tryktab i indblæsning} \label{subsub:tryktab_indblaesning}
I tabel \ref{table:oversigt_l_indblaesning}, beskriver længderne brugt til indblæsningen. 
Lufthastighed, Tryktab per meter beskrevet i tabellen er fundet ved hjælp af lindab's App `Vent Tools',
og tryktabet er herefter udregnet fra de tal.
\begin{table}[h!]
    \begin{center}
       \begin{tabular}{|l|r|r|r|r|r|}
           \hline
           Længde & meter & q$_{v}$ & V & P$_{a}$ / m & Tryktab\\
           \hline
            P & 3,94 m & 13,61 m$^3$/t & 0,19 m/s & 0,0 P$_{a}$/m & 0 Pa\\
            O & 0,74 m & 13,61 m$^3$/t & 0,19 m/s & 0,0 P$_{a}$/m & 0 Pa\\
            N & 4,71 m & 56,49 m$^3$/t & 0,78 m/s & 0,1 P$_{a}$/m & 0,471 Pa\\
            M & 4,24 m & 12,96 m$^3$/t & 0,18 m/s & 0,0 P$_{a}$/m & 0 Pa\\
            K & 3,50 m & 25,81 m$^3$/t & 0,36 m/s & 0,0 P$_{a}$/m & 0 Pa\\
            J & 1,78 m & 82,30 m$^3$/t & 1,14 m/s & 0,1 P$_{a}$/m & 0,178 Pa\\
            I & 0,74 m & 82,30 m$^3$/t & 1,14 m/s & 0,1 P$_{a}$/m & 0,074 Pa\\
           \hline
       \end{tabular}
   \end{center}
   \caption{Oversigt over længderne brugt i indblæsning}
   \label{table:oversigt_l_indblaesning}
\end{table}
Tryktabet i T-stykkerne er beskrevet i tabel \ref{table:oversigt_tryktab_t-roer_ind}. 
Lufthastighed for Stue/Alrum/Køkken og Værelse 1 er beregnet ud fra \ref{eqn:omregning_vs_til_lh} på side \pageref{eqn:omregning_vs_til_lh}.
Stue's udregning kan se i udregning \ref{eqn:udregning_af_vs_stue}.
\begin{align} \label{eqn:udregning_af_vs_stue}
    V     &= \frac{ q_v \cdot 4 }{ d_{n}^{2} \cdot \pi \cdot 3600 } \nonumber \\
    V     &= \frac{ 42,88 \cdot 4 }{ (160/1000)^{2} \cdot \pi \cdot 3600 } \nonumber \\
    V     &= \frac{ 171,53 }{ 92,16 \cdot \pi } \nonumber \\
    V     &= 0,592 \text{ m/s}
\end{align}
Når der aflæses i databladet for T-stykkerne, så er det udenfor skalaen.
Det kommer af, at rørdiameteren er sat efter udsugning det sætter lufthastigheden ned og derfor kan tabet sættes til 0 Pa.
Det betyder, at indblæsningen ikke ville have noget tab af betydning i T-stykkerne.
\begin{table}[h!]
    \begin{center}
       \begin{tabular}{|l|r|r|r|}
           \hline
           T-rør & V$_{1}$ & V$_{2}$ & Tryktab \\
           \hline
           T$_{\text{J->N}}$ & 1,14 m/s & 0,78 m/s & 0 Pa \\ 
           T$_{\text{J->K}}$ & 1,14 m/s & 0,36 m/s & 0 Pa \\ 
           T$_{\text{N->O}}$ & 0,78 m/s & 0,19 m/s & 0 Pa \\ 
           T$_{\text{N->Stue}}$ & 0,78 m/s & 0,59 m/s & 0 Pa \\ 
           T$_{\text{K->M}}$ & 0,36 m/s & 0,18 m/s & 0 Pa \\ 
           T$_{\text{K->Værelse 1}}$ & 0,36 m/s & 0,71 m/s & 0 Pa \\ 
           \hline
       \end{tabular}
   \end{center}
   \caption{Oversigt tryktabet i T-rør for indblæsning}
   \label{table:oversigt_tryktab_t-roer_ind}
\end{table}
Nu kan tabet findes på den længeste strækning. Som er på indblæsning er soveværelset.
\begin{table}[h!]
    \begin{center}
       \begin{tabular}{lcr}
           \hline
           \hline
           \textbf{Soveværelse} &  & \\
           \hline
           \hline
           Ventil (6mm åbning) & : & 15,000 Pa \\
           90$^\circ$ BU    & : & 0 Pa \\
           Rør$_{\text{P}}$ & : & 0 Pa \\
           90$^\circ$ BU    & : & 0 Pa \\
           Rør$_{\text{O}}$ & : & 0 Pa \\
           T-Stykke$_{\text{N->O}}$  & : & 0 Pa\\
           Rør$_{\text{N}}$ & : & 0,471 Pa \\
           T-Stykke$_{\text{J->N}}$  & : & 0 Pa\\
           Rør$_{\text{J}}$ & : & 0,178 Pa \\
           90$^\circ$ BU    & : & 3,800 Pa \\
           Rør$_{\text{I}}$ & : & 0,074 Pa \\
           \hline
           Samlet tryktab & : & \underline{\underline{ 19,523 Pa}} 
       \end{tabular}
   \end{center}
   %\caption{Oversigt tryktabet i T-rør}
   %\label{table:oversigt_tryktab_t-roer}
\end{table}

\subsubsection{Valg af ventilationsaggregater}
\todo{Valgt Beskrive hvordan der skal vælgees}