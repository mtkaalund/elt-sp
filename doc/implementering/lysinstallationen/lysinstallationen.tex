\subsection{Intelligent bygningsinstallation}

\subsubsection{Scenarie i IHC installationen} \label{subsub:ihc_scener}
%\todo{Beskrivelse af de forskellige scenarier}

\paragraph{Velkomst}
Ved et enkelt tryk på kontakten i bryggers, ville lyset tænde ud i bryggers.\\
Ved et langt tryk på samme kontakt, ville lyset tænde, 
samt ganglyset og lyset over køkkenbordet.\\
Efter 2 minutter, slukkes lyset i bryggeres og efter yderlig et minut slukkes lyset i gangen.

\paragraph{På vej ud af huset, i bryggers}
Ved et kort tryk slukkes alt lyset i huset undtagen værelserne. 
Lyset i bryggers slukkes efter 2 minutter. \\
Ved et langt tryk slukkes alt lyset i huset samt værelserne. 
Lyset i bryggeres slukkes efter 2 minutter, 
ventilationsanlægget startes på høj udsugning efter 10 minutter og køre efter tidsstyringen. 

\paragraph{Sluk alt lys i huset}
Dette er et scenarie, som ikke er koblet til en kontakt men i stedet bruges af andre scenarie til, 
at slukke for alt i huset med undtagelse af værelserne og soveværelset.

\paragraph{Godnat scenarie i soveværelse}
Ved et kort tryk på kontakten ved sengen, 
så slukkes lyset soveværelset og i resten af huset med undtagelse af værelserne.\\
Ved et langt tryk på kontakten ved sengen, 
så slukkes alt lyset i huset også i alle værelserne.

\paragraph{Godnat scenarie i resterende værelse}
Ved et kort tryk på kontakten ved sengen, 
så slukkes lyset soveværelset og i resten af huset med undtagelse af andre værelser og soveværelset. \\
Ved et langt tryk på kontakten ved sengen, 
så slukkes alt lyset i huset også i alle værelserne samt soveværelset.

\paragraph{Tidsstyring af ventilationsanlægget}
I hverdage ville anlægget blive startet på høj udsugning kl 10.00 hvis ud af huset ikke er blevet aktiveret inden. 
Ventilationsanlægget ville køre i en 1 time hvor efter den ville køre ned på lav hastighed. \\
Her efter ville anlægget startes på høj udsugning kl 13.00 (efter frokost) køre en halv time hvorefter den igen går ned på lav hastighed.
Igen kl 19 ville anlægget starte på høj udsugning (efter aftensmaden) køre en halv time på høj og gå ned på lav hastighed.

\paragraph{Overstyring af ventilationsanlægget}
Hvis luftfugtigheden på badværelset bliver over 30\%, så startes ventilationsanlægget på høj udsugning til luftfugtigheden er faldet til 20 \% igen. \\
Ved at trykke på den nederest kontakt til højre i køkkenet, ville ventilationsanlægget start på høj udsugning. Ved at trykke på samme kontakt ville anlægget stoppe. \\
Ved at trykke på den nederest kontakt til højre i bryggers, ville ventilationsanlægget start på høj udsugning. Ved at trykke på samme kontakt ville anlægget stoppe.

%\subsubsection{Styring af udsugning i drivhuset} \label{subsub:ihc_drivhus}
%\todo{Beskrive hvordan styring af fugten i drivhuset}
%\todo{Hvis fugtigtheden falder til under 60 \%, starter styringen. Hvis den er over 80 \% stopper styringen}
%Da IKEA's Trådfri app kun kan lave den mest basale styringen og ikke det som kunden til efterspørger. Så er vi ``tvunget`` til at kigge på nogle lidt andre løsningen end de meste standarde løsningen.

%\subsubsection{Home-Assistant}
%Home-Assistant (\cite{HAW:2018:Online}) gør det muligt, at samle flere typer af smart bygningsinstallationer under en styring.
%Det er muligt, at bruge IKEA Trådfri sammen med en IHC styring (det kræver dog, at IHC er på netværket). Den understøtter flere forskellige systemer og bliver hele tide udvidet med nye systemer.
%Varmestyringen kan også integeres i denne løsningen, samt egne udviklet enheder kan også blive integret sammen med den. 
%Dette system kan installeres på en minicomputer så for eksemple en Raspberry Pi 3. Når hele systemet er sat op med de enheder som er i installationen, så kan det tilgåes fra en hjemmeside som i modsætning af IHC controlleren ikke er lavet af JAVA.
%Her er det valgt at bruge en Raspberry Pi 3 model b+, og det er den billed-fil af hass.io som også er blevet hentet ned.