\subsection{Intelligent bygningsinstallation}
Til den nye udvidelse er der valgt, at bruge en IHC installation med en IHC controller. Planen er, at når resten af husets el installation bliver moderniseret, at det også bliver tilsluttet IHC controlleren.
Men her i første omgang, så er det kun i den nye tilbygning som IHC controlleren har fat i. I den ``gamle`` installation, så bliver lysinstallationen styret fra IKEA Trådfri enheder.

Da IKEA's Trådfri app kun kan lave den mest basale styringen og ikke det som kunden til efterspørger. Så er vi ``tvunget`` til at kigge på nogle lidt andre løsningen end de meste standarde løsningen.

\subsubsection{Home-Assistant}
Home-Assistant (\cite{HAW:2018:Online}) gør det muligt, at samle flere typer af smart bygningsinstallationer under en styring.
Det er muligt, at bruge IKEA Trådfri sammen med en IHC styring (det kræver dog, at IHC er på netværket). Den understøtter flere forskellige systemer og bliver hele tide udvidet med nye systemer.

Varmestyringen kan også integeres i denne løsningen, samt egne udviklet enheder kan også blive integret sammen med den. 

Dette system kan installeres på en minicomputer så for eksemple en Raspberry Pi 3. Når hele systemet er sat op med de enheder som er i installationen, så kan det tilgåes fra en hjemmeside som i modsætning af IHC controlleren ikke er lavet af JAVA.